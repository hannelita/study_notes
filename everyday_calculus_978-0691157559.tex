\documentclass{cornell}
\usepackage{lipsum}
\begin{document} 

\title{
    \vspace{-3em}
        \begin{tcolorbox}[colframe=white,opacityback=0]
            \begin{tcolorbox}
                \Huge\sffamily "Everyday Calculus", FERNANDEZ, Oscar E., 2014, Princeton University Press
            \end{tcolorbox}
        \end{tcolorbox}
    \vspace{-3em}
}
\maketitle

\begin{tcolorbox}
\preread{I read this book for fun, in 2015. I was looking for a light introduction to Calculus.}{Is there a good way to get started with the concepts of limits, derivatives, ...?}
\end{tcolorbox}


\noindent
This book is a narrative of a Maths professor. He points many mathematical concepts that he notices in his daily life. Meanwhile, he describes his routine, since the time he wakes up until he goes to bed. Some of the descriptions are a little bit exaggerated, but the author does a good job fitting the concepts of functions, limits, derivatives and integrals in a conventional narrative.

%\begin{tcolorbox}[colback=white,opacityframe=0]
\topic{Ch. 1 - Functions}%
{Sleep cycles take 90 minutes. Sleep 7.5 hours. Express the stages of sleep as a trigonometric function.}%
{You need electricity if you have a digital clock. The author mentions T. Edison and M. Faraday and explains the difference between AC/DC, using the context to show different types of functions. Next, we find stories about logarithms and trigonometric functions.}%
%\end{tcolorbox}

\topic{Ch. 2 - Introduction to Calculus}%
{The author describes his breakfast time and mentions an example with the rate of temperature change for a cup of coffee, followed by the formal definition of the derivative. We can find many examples  in the next pages, including AROCs and vitamins. Derivatives are about change.}%

\topic{Ch. 3 - Derivatives}%
{A bit of history - Galileo's attempts to describe the motion with a formula. Examples with the speed of a raindrop. Finding the graphical representation of the derivative (tangent lines). Analysis of the second derivative. Examples of road accidents, unemployment and population.}%


\topic{Ch. 4 and 5 - More about differentiation}%
{Analysis of the pouring ratio to fill a cup of hot chocolate. Interesting example of blood vessels radius and the resistance of a liquid travelling through it (mathematics in the human body). Solving optimization problems with calculus (min and max).}%

\begin{tcolorbox}
\insight{\( AvgProduction(x) = \frac{p(x)}{x} \). Take the derivative: \( A'(x) = \frac{xp'(x) - p(x)}{x^2} \) . \( p(x) \) needs to be \( x^2 \) or \( x^3 \) to increase the production of A. }
\end{tcolorbox}

\topic{Ch 6 - Sums}%
{A train, changing it's speed, travels a certain distance. The author presents the idea of the Riemann Sum to calculate the distance d. Next, we meet the main idea of integration ∫. There is a good use case - estimate the time you will wait for a certain event using integrals.}%

\topic{Ch 7 - Derivatives + Integrals}%
{Interesting example to determine the best seats in a movie theater}%


\summary{Short book, with interesting ideas that allow the reader to see calculus and it's applications. Downsides: Requires familiarity with the American Culture (you will find it and out of context if you have never been to Boston). Most of the examples are concrete. A few formulas seem to pop out of nowhere, but it is not a frequent scenario. }



\end{document}