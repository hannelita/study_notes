\documentclass{cornell}
\usepackage{lipsum}
\begin{document} 

\title{
    \vspace{-3em}
        \begin{tcolorbox}[colframe=white,opacityback=0]
            \begin{tcolorbox}
                \Huge\sffamily "Seventeen Equations that Changed the World", STEWART, Ian, 2013, Zahar
            \end{tcolorbox}
        \end{tcolorbox}
    \vspace{-3em}
}
\maketitle

\begin{tcolorbox}
\preread{I read the Portuguese(Brazilian) version of this book, in 2016. I was looking for inspiration.}{I want to know what Ian Stewart classifies as 'an important equation that changed the world'.}
\end{tcolorbox}


\noindent
This book has seventeen chapters (as expected). 17 is a prime number (hehe). Ian Stewart is very good at telling stories, so each chapter teaches you about the context that lead to that particular equation, people who had discovered it previously, the lucky person who got the credit for the work, applications of the equation (direct and indirect applications) and the fields of mathematics that it created (or helped to do so).

%\begin{tcolorbox}[colback=white,opacityframe=0]
\topic{Ch. 1 - Pythagoras theorem}%
{I did not learn much from this chapter, since I was familiar with Math history}%
%\end{tcolorbox}

\topic{Ch. 2 - Logarithms}%
{Not much new}%

\topic{Ch. 3 - Calculus}%
{My side notes started at this chapter. Limits appeared after the study of derivatives (Maybe that's why students tend to consider it more difficult than calculating derivatives).}%
{Names that I want to search more about: Bernard Bolzano, Karl Weierstrass}%

\topic{Ch. 4 - Newton's Law of Gravity}%
{From Copernicus to Poincaré. Hohmann transfer orbits. Lagrange points. Tubes in the space}%

\topic{Ch. 5 - Square Root of -1}%
{Why are the squares always positive numbers?}%
{Names that I want to search more about: Wallis}%

\topic{Ch. 6 - Euler's Formula for Polyhedra}%
{ \( F-E+V=2 \) . Invariants and Topology. Gauss and Riemann. Compact surfaces(no border, finite extension). Knots. Jones polynomials. Homfly Polynomials. Topology and the DNA}%
{Names that I want to search more about: Homfly}%

\topic{Ch. 7 - Normal Distribution}%
{Quetelet. IThis chapter reminded of the book "Lady Tasting Tea", from D. Salsburg, 2001. Be careful with the null hypothesis.}%


\topic{Ch. 8 - Wave Equation}%
{Good explanation of partial derivatives. Linear combinations: if u(x,t) and v(x,t) are solutions, then any au(x,t)+bv(x,t) are solutions; a and b are constants.}% 
{The equation comes from Newton's Second Law of motion (F = ma). Laplacian \( \nabla^2 \) average difference between u in a point and its neighbourhood. \( \nabla^2 \) is the wave equation for 3d. Bernoulli. Applications: Earthquakes.}%

\topic{Ch. 9 - Fourier transform}%
{Pattern in space and time -> sin and frequencies. Time -> frequency Heat != mechanical waves. The mathematical instrument must be different as well. Infinite sums. One more mention to \( \nabla^2 \). What is the definition of a function? Fourier had trouble with that. Cantor. Series converges as a whole, not from point to point (wow!). JPEG. Wavelets and .wsa}% 
{Names that I want to search more about: Henri Lebesgue}%

\topic{Ch. 10 - Navier-Stokes Equation}%
{It is like Newton's Second Law of Motion. Building an aircraft (funny remark: it does not mention the Brazilian creator Santos-Dumont). Applications at fluid mechanics. Extensive use of vector calculus. Modelling blood transport with tubes. }%

\topic{Ch. 11 - Maxwell's Equations}%
{Unification - electricity and magnetism. M. Faraday's short story. Maxwell borrows the main ideas of Fluid Mechanics. Electricity and magnetism are 'not compressible': they can't flow out to nowhere (divergence operator \( \nabla \cdot E = 0\) and \( \nabla \cdot H = 0 \) ) If one of the fields moves in a circle, a perpendicular force appears ( Curl operator; \( \nabla \times E \) and \( \nabla \times H \)) . }%
{We can obtain the wave equation from Maxwell's equations. Tesla's story. Discussions about the behaviour of the light. Maxwell's Equations did not change the world, they opened the doors to a new one.}%
{Remark: Stweart dedicates more pages to the subjects he knows more about. J. Marey and G.H Hardy made the throwing cat experiment. How many cats did they throw before figuring out the torsion process of the cat's body? }%
{Names that I want to search more about: Claude Chappe, Oliver Lodge, A. Muirhead, G. Marconi, W. Shockley, W. Brattain, J. Bardeen.}%

\begin{tcolorbox}
\insight{Can I use Maxwell ideas to homotopy type theory? 'the function does not flow; some action creates an orthogonal side effect, expressing it with the curl operator'}
\end{tcolorbox}

\topic{Ch. 12 - Second law of thermodynamics}%
{The entropy always increases. Entropy is like the heat - we define it in terms of state change, not as a single state. Maxwell-Boltzmann. Interpreting the entropy as a macrostate and obtaining traces of the microstates - interpreting it as the lack of order. Brownian movement. }%
{Boltzmann committed suicide :(. Paradoxes of interpreting entropy as disorder. I read about entropy and its application for black holes at S. Hawking's book, "A Brief History of time".  )}%
{Names that I want to search more about: C. P. Snow., T. Thiele, M. Smoluchoeski, J. B. Perrin, E. Schrodinger}%

\topic{Ch. 13 - Relativity}%
{Not much new here, I had read "The Evolution of Physics", from A. Einstein, a few weeks ago. Light cones. Schwarzschild metrics and singularities. Wimps (particles). Can we test a theory that foresees anything? Mond theory. Discussion about the Big Bang. GPS need the relativity theory!}%
{Names that I want to search more about: H. Lorenz, U. Le Verrier, F. Zwicky, M. Milgrom, J. Smoller, Blake Temple, R. McKay, Colin Rourke, Halton Arp.}%


\topic{Ch. 14 - Schrödinger's Equation}%
{Interpreting the matter as a wave. Quantum mechanics. Max Planck and G. Kirchhoff. The nature is discrete in small scales. Photons (light as a particle). Light's duality. W. Heisenberg. Schrödinger's Equation is a differential equation. We need the complex domain, since the wave function has some sort of phase (i represents this phase, just like in Electrical Eng. subjects). Autovalues (eigenvalues). Hamiltonian operator. Eigenvalues -> shift in time. Copenhage interpretation. Schrödinger's cat. Many-World interpretation. Decoherence and it's obstacles to quantum computers (it destroys the superposition).}%
{Names that I want to search more about: Lord Kelvin, De Broglie, H. Everett Jr., Bryce DeWitt}%


\topic{Ch. 15 - Information Theory}%
{How much information there is in a message. Mathematical theory of communications. How much information can we transport in a channel? Coding theory. Multidimensional geometry. Associate sequences with the vertexes of a hypercube; for a curved space, we use Riemann's metrics (is topology useful here?). Abstract algebra. Hamming code. Galois fields (operation set with a finite number of elements). Reed-Solomon code. Galois influenced the electronics! }%
{Applications - electronics, bio-informatics, DNA. Another definition for entropy.}%
{Names that I want to search more about: Edwin Jaynes}%


\topic{Ch. 16 - Chaos Theory}%
{Short chapter. Deterministic models (knowing information about the present determines the future). }%
{Names that I want to search more about: Robert May}%

\topic{Ch. 17 - Black-Scholes Equation}%
{Another short chapter. Profit and financial market. It is a PDE (partial differential equation), second order}%
{Author: "...The money comes in discrete packages"; me: "quanta?" :P }%

\summary{Pleasant, informative book. Go for it; don't be afraid of the equations.}



\end{document}